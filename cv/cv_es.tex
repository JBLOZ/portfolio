%-------------------------
% Currículum en Latex
% Autor : Jake Gutierrez
% Basado en: https://github.com/sb2nov/resume
% Licencia : MIT
%------------------------

\documentclass[letterpaper,11pt]{article}

\usepackage{latexsym}
\usepackage[empty]{fullpage}
\usepackage{titlesec}
\usepackage{marvosym}
\usepackage[dvipsnames]{xcolor}
\usepackage{verbatim}
\usepackage{enumitem}
\usepackage[hidelinks]{hyperref}
\usepackage{fancyhdr}
\usepackage[english]{babel}
\usepackage{tabularx}
\usepackage{fontawesome5}
\usepackage{multicol}
\usepackage[most]{tcolorbox}
\setlength{\multicolsep}{-3.0pt}
\setlength{\columnsep}{-1pt}

\pagestyle{fancy}
\fancyhf{} % clear all header and footer fields
\fancyfoot{}
\renewcommand{\headrulewidth}{0pt}
\renewcommand{\footrulewidth}{0pt}

% Adjust margins
\addtolength{\oddsidemargin}{-0.6in}
\addtolength{\evensidemargin}{-0.5in}
\addtolength{\textwidth}{1.19in}
\addtolength{\topmargin}{-.7in}
\addtolength{\textheight}{1.4in}

\urlstyle{same}

\raggedbottom
\raggedright
\setlength{\tabcolsep}{0in}

\titleformat{\section}{
  \vspace{-10pt}\scshape\raggedright\large\bfseries
}{}{0em}{}[\color{black}\titlerule \vspace{-10pt}]

%-------------------------
% Custom commands
\newcommand{\resumeItem}[1]{
  \item\small{
    {#1 \vspace{-2pt}}
  }
}

\newcommand{\classesList}[4]{
    \item\small{
        {#1 #2 #3 #4 \vspace{-2pt}}
  }
}

\newcommand{\resumeSubheading}[4]{
  \vspace{-2pt}\item
    \begin{tabular*}{1.0\textwidth}[t]{l@{\extracolsep{\fill}}r}
      \textbf{#1} & \textbf{\small #2} \\
      \textit{\small#3} & \textit{\small #4} \\
    \end{tabular*}\vspace{-7pt}
}

\newcommand{\resumeSubSubheading}[2]{
    \item
    \begin{tabular*}{0.97\textwidth}{l@{\extracolsep{\fill}}r}
      	extit{\small#1} & \textit{\small #2} \\
    \end{tabular*}\vspace{-7pt}
}

\newcommand{\resumeProjectHeading}[2]{
    \item
    \begin{tabular*}{1.001\textwidth}{l@{\extracolsep{\fill}}r}
      \small#1 & \textbf{\small #2}\\
    \end{tabular*}\vspace{-7pt}
}

\newcommand{\resumeSubItem}[1]{\resumeItem{#1}\vspace{-7pt}}

\newcommand{\bluehref}[2]{\href{#1}{\textcolor{RoyalBlue}{#2}}}

\renewcommand\labelitemi{$\vcenter{\hbox{\tiny$\bullet$}}$}
\renewcommand\labelitemii{$\vcenter{\hbox{\tiny$\bullet$}}$}

\newcommand{\resumeSubHeadingListStart}{\begin{itemize}[leftmargin=0.0in, label={}]}
\newcommand{\resumeSubHeadingListEnd}{\end{itemize}}
\newcommand{\resumeItemListStart}{\begin{itemize}}
\newcommand{\resumeItemListEnd}{\end{itemize}\vspace{-3pt}}

\newenvironment{resumeSection}[1]{
  \begin{tcolorbox}[
    colback=white,
    boxrule=0pt,
    sharp corners,
    enhanced,
    breakable,
    left=4pt,
    right=4pt,
    top=4pt,
    bottom=4pt,
    borderline west={0.6pt}{0pt}{black},
    borderline east={0.6pt}{0pt}{black},
    borderline north={0.6pt}{0pt}{black},
    borderline south={0.6pt}{0pt}{black}
  ]
  {\scshape\large\bfseries #1}\par\vspace{1pt}
}{
  \end{tcolorbox}
  \vspace{-2pt}
}

%-------------------------------------------
%%%%%%  RESUME STARTS HERE  %%%%%%%%%%%%%%%%%%%%%%%%%%%%

\begin{document}

\begin{center}
    {\Huge \scshape Jordi Blasco Lozano} \\ \vspace{1pt}
    Alicante, Espa\~na \\ \vspace{1pt}
    \small \raisebox{-0.1\height}\faPhone\ +34\,622\,47\,05\,35 ~
    \href{mailto:jordiblloz@gmail.com}{\raisebox{-0.2\height}\faEnvelope\ \underline{jordiblloz@gmail.com}} ~
    \href{https://linkedin.com/in/jbloz}{\raisebox{-0.2\height}\faLinkedin\ \underline{linkedin.com/in/jbloz}} ~
    \href{https://github.com/JBLOZ}{\raisebox{-0.2\height}\faGithub\ \underline{github.com/JBLOZ}}
\end{center}

\vspace{4pt}
{\scshape\large\bfseries Perfil}\par
\vspace{2pt}
\small Soy un estudiante de 3er año de Ingeniería en Inteligencia Artificial en busca de oportunidades para aplicar y ampliar mis conocimientos en IA, especialmente en proyectos relacionados con Python, algoritmos y desarrollo de aplicaciones. Poseo una sólida formación en inteligencia artificial, programación y algoritmos, junto con experiencia práctica en el desarrollo de videojuegos y colaboración internacional.
\vspace{6pt}

%-----------INTERNATIONAL MOBILITY-----------
\begin{resumeSection}{Movilidad Internacional y Proyectos Clave}
  \resumeSubHeadingListStart
    \resumeProjectHeading
      {\textbf{Erasmus Home Human Out Migration in Europe} $|$ \emph{Budapest, Hungría}}{Sept. 2021}
      \resumeItemListStart
        \resumeItem{Colaboré con socios internacionales en la iniciativa Erasmus \bluehref{https://erasmus-plus.ec.europa.eu/projects/search/details/2020-1-FR01-KA229-079855}{HOME} que aborda la emigración humana en Europa.}
      \resumeItemListEnd
    \resumeProjectHeading
      {\textbf{Proyectos Universitarios} $|$ \emph{Alicante, España}}{2023 -- Actualidad}
      \resumeItemListStart
  \resumeItem{Implementé un \bluehref{https://github.com/JBLOZ/ROBOT-fuzzy-and-basiclogic}{sistema difuso} en Python usando \texttt{fuzzy\_expert} para controlar que un robot 2D siguiera lineas.}
  \resumeItem{Desarrollé y documenté un \bluehref{https://github.com/JBLOZ/CNN-classification----geometric2d-and-cancer-dangerousness}{modelo ResNet-50} en MATLAB para la detección de cáncer de piel, alcanzando una precisión del 80\% para clasificaciones cancerosas y del 85\% para no cancerosas.}
  \resumeItem{Construí una \bluehref{https://github.com/JBLOZ/py-ecommerce-software-proyect}{aplicación web de comercio electrónico inteligente} que permite la búsqueda de productos por texto o imagen a través de modelos de IA preentrenados.}
  \resumeItem{Creé un \bluehref{https://github.com/JBLOZ/heart-attack-fuzzy-system}{sistema experto difuso} para estimar el riesgo de ataque cardíaco basado en preguntas clínicas estructuradas.}
  \resumeItem{Diseñé un \bluehref{https://github.com/JBLOZ/PACMAN-CNN-Alphabeta}{agente de Pacman} que combina una CNN con heurísticas Alfa-Beta para un juego estratégico.}
  \resumeItem{Desarrollé y evalué un sistema de clasificación multiclase de animales, comparando el rendimiento de clasificadores como Bayes Gaussiano, ventanas de Parzen y variantes de k-NN en un \bluehref{https://jbloz.github.io/portfolio/proyectos/aprendizaje_automatico/p_02/memoria.pdf}{conjunto de datos balanceado del Zoo con 7 especies}.}\resumeItem{Diseñé e implementé un \bluehref{https://jbloz.github.io/portfolio/proyectos/aws/p_03/memoria_03.pdf}{despliegue híbrido y seguro en AWS}, cubriendo la segmentación de VPC, el endurecimiento de políticas de S3 y el aprovisionamiento automatizado de EC2 para front/back, junto con un \bluehref{https://jbloz.github.io/portfolio/proyectos/aws/p_04/memoria_04.pdf}{despliegue canary avanzado} que escala las API de recomendación a través de subredes, grupos de destino de ALB y actualizaciones con containers.}\resumeItem{Para estudios de caso adicionales, explore mi \bluehref{https://jbloz.github.io/portfolio/portfolio/index.html}{portafolio de proyectos}.}
      \resumeItemListEnd
  \resumeSubHeadingListEnd
\end{resumeSection}

%-----------INDI PROJECTS-----------
\begin{resumeSection}{Proyectos Personales}
  \resumeSubHeadingListStart
    \resumeProjectHeading
      {\textbf{Análisis de Spotify Wrapped} $|$ \emph{Alicante, España}}{Oct. 2024 -- 2025}
      \resumeItemListStart
  \resumeItem{Desarrollé un cuaderno de Jupyter que ofrece un análisis en profundidad de los \bluehref{https://github.com/JBLOZ/Spotify_Stats}{hábitos de escucha de Spotify.}}
      \resumeItemListEnd
    \resumeProjectHeading
      {\textbf{Videojuego 3D con Unreal Engine 4} $|$ \emph{Alicante, España}}{2022}
      \resumeItemListStart
  \resumeItem{Monté y programé un \bluehref{https://www.youtube.com/watch?v=VaXI3NCUENQ}{videojuego 3D} utilizando Blueprints de Unreal Engine y C++.}
      \resumeItemListEnd
  \resumeSubHeadingListEnd
\end{resumeSection}

%-----------PROFESSIONAL EXPERIENCE-----------
\begin{resumeSection}{Experiencia Profesional}
  \resumeSubHeadingListStart
    \resumeProjectHeading
      {\textbf{Outlier} $|$ \emph{Alicante, España}}{2024 -- Actualidad}
      \resumeItemListStart
        \resumeItem{Trabajando como Investigador de Aprendizaje por Refuerzo a partir de Retroalimentación Humana, contribuyendo a múltiples proyectos de aprendizaje por refuerzo impulsados por retroalimentación humana.}
      \resumeItemListEnd
  \resumeSubHeadingListEnd
\end{resumeSection}

%-----------EDUCATION-----------
\begin{resumeSection}{Educación}
  \resumeSubHeadingListStart
    \resumeSubheading
      {Universidad de Alicante}{Sept. 2022 -- Actualidad}
      {Grado en Ingeniería en Inteligencia Artificial (3er año, 100\% de asignaturas superadas)}{Alicante, España}
    \resumeSubheading
      {Oxford Test of English}{Mayo 2024}
      {Nivel B2, ref 263339}{Alicante, España}
  \resumeSubHeadingListEnd
\end{resumeSection}

%-----------ADDITIONAL SKILLS-----------
\begin{resumeSection}{Habilidades Adicionales}
  \resumeSubHeadingListStart
    \item[]\small
      	\textbf{Herramientas:} AWS, Latex, git, Docker, Bash \\
      	\textbf{Herramientas de Machine Learning:} PyTorch, Hugging Face, pandas, numpy, matplotlib, n8n \\
      	\textbf{Programación:} Python, C++, C, Swift, Matlab, noSQL, SQL, JavaScript \\
      	\textbf{Idiomas:} Inglés (Fluido), Español (Nativo), Catalán (Nativo)
  \resumeSubHeadingListEnd
\end{resumeSection}

\end{document}
